%***********************************************%
%												%
% EECS 470 - Project 2							%
%<------------------> 							%
% Last Modified by:								%
%	William Cunningham on 2014-9-14				%
%												%
%***********************************************%

%***********************************************%
% Preamble										%
%***********************************************%

\documentclass{article}

\usepackage[dvipsnames]{xcolor}

\usepackage{
	tikz,
	colortbl,
	graphicx,
	amsmath,
	amssymb,
	mathrsfs,
	float,
	siunitx,
	fancyhdr,
	url,
	minted,
	cleveref,
	algorithm,
	algpseudocode
}


\usepackage
	[left=1in,top=1in,right=1in,bottom=1in]
	{geometry}

\pagestyle{fancy}

%--- Header ---%

\newcommand{\courseNumber}{EECS 470}
\newcommand{\courseTitle}{Computer Architecture}
\newcommand{\university}{University of Michigan, Ann Arbor}
\newcommand{\labdate}{January 30$^{\text{th}}$, 2021}


\lhead{
	\small{
		\university
	}
}
\rhead{
	\small{
		\emph{Date: \labdate} \hspace*{-1em}
		% Why is the above \hspace necessary?
	}
}

\newcommand{\shortbar}{
	\vspace*{-12pt}
	\begin{center}
		\rule{5ex}{0.1pt}
	\end{center}
}
\newcommand{\project}[1]{
	\begin{center}
		\LARGE{
			\vspace*{-12pt}
			EECS 470 Project \##1
			\shortbar
		}
	\end{center}
}


%***********************************************%
%                                               %
% TikZ Definitions                              %
%                                               %
%***********************************************%

\usetikzlibrary{shapes,arrows,automata,shadows}
\pgfdeclarelayer{background}
\pgfdeclarelayer{foreground}
\pgfsetlayers{background,main,foreground}

% Block Diagram Styles
\tikzstyle{block} = [draw, fill=RoyalBlue!50, rectangle,
        minimum height=2cm, minimum width=2cm, rounded corners]
\tikzstyle{sum} = [draw, fill=blue!20, circle, node distance=2cm]
\tikzstyle{input} = [coordinate]
\tikzstyle{output} = [coordinate]
\tikzstyle{branch} = [coordinate]
\tikzstyle{pinstyle} = [pin edge={to-, thin, black}]

% Signal Flow Graph Styles
\tikzstyle{signal} = [draw, fill=RoyalBlue!50, circle,
	minimum height=3em]
\tikzstyle{state} = [draw, fill=RoyalBlue!50, circle,
	minimum height=3em]

%***********************************************%
%                                               %
% Document										%
%                                               %
%***********************************************%

\begin{document}
\vspace*{-20pt}
\project{2}

\begin{itemize}
	\item This is an individual assignment. You may discuss the specification
		and help one another with the (System)Verilog language. Your solution,
		particularly the designs you submit, must be your own.
	\item Due at 11:59pm ET on 10$^{\text{th}}$ February, 2021.
		\emph{Late submissions are not accepted.}
\end{itemize}
\hrulefill

\section{Introduction}
We've mentioned synthesis several times already in the lab and lecture.
Synthesis is an exponentially complex process, and so it can take a very long
time on larger projects, like the one you'll be doing at the end of the
semester. To mitigate this, we can synthesize submodules and then include them
as black boxes in the final project, which simplifies the synthesis problem.
This is particularly useful for synthesizing the caches in your final project,
which can reduce the synthesis time by half or more. 

We will use a pipelined multiplier as an example of this process. This
multiplier is actually the one you'll be using in your final project, so you 
should use this as a chance to get very familiar with it. 

Finally, we will be using this multiplier as part of a finite state machine
design problem, which is intended to give you additional practice writing
Verilog in reasonable style. The style we recommend for finite state machine
design will be presented in the lab.

\section{Hierarchical Synthesis}
\subsection{Concept}
Synthesis takes a behavioral level design and turns it into a structural level
design. Specifically, the synthesis tool replaces larger operations, i.e.
multiplication or addition with standard designs for modules, and attempts to
build larger logic components out of a set of standard cells (our standard cells
are in the \texttt{lec25dscc25.v} file included by your \texttt{Makefile}).
Doing this optimally is an NP-hard problem, so it is often impossible in 
practice. In this class, it is generally merely very time consuming, meaning 
that your final project will take anywhere between six hours and one day to 
synthesize. 

To mitigate this problem, we can synthesize large submodules in a design
individually and then include these syntheses as black boxes in the final design
synthesis. Given that the original problem was NP-hard, we know that it would
have been exponentially complex, where the quantity of interest is the size of a
design, measured in something like the number of logic elements (standard cells,
LUTs, transistors, etc.) That means that simplifying to only somewhat fewer
elements still shortens the time to find a solution significantly.

\subsection{Pipelined Multiplier}
We have provided you with a pipelined multiplier, found in 
\texttt{pipe\_mult.v}, to which we will apply this concept. The multiplier does 
multiplication in stages, somewhat like you would have learned to carry out 
multiplication in elementary school. It multiplies the first 8 bits of the
multiplier with the whole multiplicand in one clock cycle, then the next 8 bits
of the multiplier against a shifted multiplicand, and so on to get 8 partial
products. Summing those gives us the desired multiplication. This means that 
each multiplication will take 8 clock cycles. Each partial product is created by
a separate multiplier stage, which can be found in the  \texttt{mult\_stage.v} 
file. 

Your first assignment will be to find a reasonable clock period (within
2\si{\nano\second} of the lowest possible) at which the 8 stage multiplier we've
provided you can be synthesized with slack met. For example, if you find that
the design does not synthesize at a clock period of 4 \si{\nano\second}, but it 
does synthesize with a clock period of 5.5 \si{\nano\second}, then 5.5 
\si{\nano\second} must be within 1.5 \si{\nano\second} of the shortest possible 
clock period. We've already setup the \texttt{.tcl} scripts to synthesize the 
two modules in this design for you. Note the  \texttt{set\_dont\_touch} command 
in the \texttt{mult.tcl} script. This command tells the synthesis tool to treat 
the module as a black box and to not optimize it further. 

To find the clock period, you will need to change the clock period in the
\texttt{mult\_stage.tcl} file until you reach the best clock period that meets
all slack. Once you've found that, the total clock period of the multiplier will
be that stage clock period plus the additional combinational delay from the
interconnects in the higher level \texttt{mult} module. This means that the
clock period in the \texttt{mult\_stage.tcl} file will be different from the one
found in the \texttt{mult.tcl} file.

Your second assignment is to modify the pipelined multiplier we've provided to
work as both a 4 stage multiplier and as a 2 stage multiplier. You can either
do this by copying the files and having separate 2, 4 and 8 stage multipliers or
by figuring out a combination of preprocessor macros or parameters that set
pipeline depth. Whichever method you choose is fine for this project, but having
the parameterized pipeline depth will be extremely helpful for the final
project. Once you have the other two multipliers, you will need to find the best
clock periods that they can each achieve. How do all three clock periods
compare? Does this conform to your expectations?

\section{Integer Square Root} 
\label{sec:isr}
Now, you will need to create a module that uses the multiplier that we supplied,
the 8 stage multiplier. You will be writing a module to compute the integer
square root of a 64-bit number. It will generate a 32-bit number that is the
largest integer that is not larger than the square root of the number provided.
For example, the integer square root of 24 is 4.

The module declaration is as follows:
\begin{figure}[H]
	\begin{minted}[frame=lines,tabsize=4]{verilog}
module ISR(
	input   				reset,
	input   		[63:0]  value,
	input   				clock,
	output logic    [31:0]  result,
	output logic            done
);
	\end{minted}
\end{figure}
It should operate as follows:
\begin{itemize}
	\item If \texttt{reset} is asserted during a rising clock edge (synchronous
		reset), the \texttt{value} signal is to be stored.
	\item If \texttt{reset} is asserted part way through a computation, the
		result of that computation is discarded and a new \texttt{value} is
		latched into the module.
	\item When the module has finished computing the answer, the output is
		placed on the \texttt{result} line and \texttt{done} line is raised on the same cycle.
	\item It must not take more than 600 clock cycles to compute a result (from
		the last clock that \texttt{reset} is asserted to the first clock that
		\texttt{done} is asserted.)
\end{itemize}

We do not suggest that you pipeline this module. You will likely need to
perform something like a binary search to find the result a simple algorithm is
as follows:
\begin{algorithm}[H]
	\caption{Integer Square Root}
	\begin{algorithmic}[1]
		\Procedure{ISR}{\texttt{value}}
		\For{$i \gets 31 \text{ to } 0$}
			\State \texttt{proposed\_solution[$i$]}$\leftarrow$1
			\If{\texttt{proposed\_solution}$^{2} >$ \texttt{value}}
			\State \texttt{proposed\_solution[$i$]}$\leftarrow$0
			\EndIf
		\EndFor
		\EndProcedure
	\end{algorithmic}
\end{algorithm}

Note that loops do not have a direct hardware equivalent. What hardware design
technique lets us implement a procedure like this?

In addition to writing this module, you will need to write a testbench for it.
This testbench should probably test specific corner cases, random testing and
the short loops. Your testbench should print either \texttt{@@@Passed} or
\texttt{@@@Failed}. 

Once you have the module written and tested, synthesize it. This will probably 
take several minutes at least.

\section{Comprehension Questions}

Answer the following questions, and submit the answers in a plaintext file called
\texttt{answers.txt} along with the rest of your project.

\begin{enumerate}
	\item Consider the multiplier supplied with this project.
		\begin{enumerate}
			\item What is the cycle time achieved when you synthesized our 
				multiplier? 
			\item What does this mean the total latency is for a multiplication?
		\end{enumerate}
	\item Answer question 1 (parts a and b) for the two multipliers you created.
	\item Consider the relative values of the answers you found to questions 1
		and 2. Do these seem reasonable? Why or why not?
	\item What is the clock period achievable with the module you wrote in
		\cref{sec:isr}?
	\item How long would it take for your module to compute the square root of 
		1001 given the cycle time of question 4? Would you expect a performance 
		gain or penalty if you used your 2 stage multiplier?
\end{enumerate}

\section{Submission}
After you are confident in your solution, make sure you have the following files
in your directory
\begin{enumerate}
	\item \texttt{ISR.v}
	\item \texttt{test\_ISR.v}
	\item \texttt{answers.txt}
\end{enumerate}
Now, go up one directory (\texttt{cd ..}) and run:

\texttt{/afs/umich.edu/user/j/b/jbbeau/Public/470submit -p2 <directory>}

\noindent
Note that the script takes a directory name as an argument, not an absolute
path, so you must run it from one level above your project 1 directory. 

At this point, you should see something similar to: 
\inputminted[frame=lines,obeytabs,tabsize=4]{bash}{submission.txt}
\noindent
If there is a problem, it will be printed to the screen. Shortly after
submission you should receive an email telling you if your submission passed the
basic tests. This is not an autograder, so it's simply telling you that your
design built successfully on the grading machine and that the module
declarations looked approximately correct. If your submission didn't pass, copy
the error message into an email to the course staff, and we will attempt to
help.

\end{document}
