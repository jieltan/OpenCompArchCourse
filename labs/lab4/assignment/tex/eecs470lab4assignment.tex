%***********************************************%
%												%
% EECS 470 - Lab 04								%
%<------------------> 							%
% Last Modified by:								%
%	William Cunningham on 2014-9-22			%
%												%
%***********************************************%

%***********************************************%
% Preamble										%
%***********************************************%

\documentclass{article}

\usepackage{
	tikz,
	xcolor,
	colortbl,
	graphicx,
	amsmath,
	amssymb,
	mathrsfs,
	float,
	siunitx,
	fancyhdr,
	url,
	listings,
	cleveref
}

\usepackage
	[left=1in,top=1in,right=1in,bottom=1in]
	{geometry}

\pagestyle{fancy}
\raggedright

%--- Header ---%

\newcommand{\courseNumber}{EECS 470}
\newcommand{\courseTitle}{Computer Architecture}
\newcommand{\university}{University of Michigan, Ann Arbor}
\newcommand{\labdate}{31$^{\text{st}}$ January, 2019}


\lhead{
	\small{
		\university
	}
}
\rhead{
	\small{
		\emph{Date: \labdate} \hspace*{-1em}
		% Why is the above \hspace necessary?
	}
}

\newcommand{\shortbar}{
	\vspace*{-12pt}
	\begin{center}
		\rule{5ex}{0.1pt}
	\end{center}
}
\newcommand{\lab}[1]{
	\begin{center}
		\LARGE{
			\vspace*{-32pt}
			EECS 470 Lab #1 Assignment
			\shortbar
			\vspace*{-20pt}
		}
	\end{center}
}


%***********************************************%
%                                               %
% TikZ Definitions                              %
%                                               %
%***********************************************%

\usetikzlibrary{shapes,arrows}

% Block Diagram Styles
\tikzstyle{block} = [draw, fill=blue!20, rectangle,
        minimum height=3em, minimum width=6em]
\tikzstyle{sum} = [draw, fill=blue!20, circle, node distance=2cm]
\tikzstyle{input} = [coordinate]
\tikzstyle{output} = [coordinate]
\tikzstyle{branch} = [coordinate]
\tikzstyle{pinstyle} = [pin edge={to-, thin, black}]

% Signal Flow Graph Styles
\tikzstyle{signal} = [draw, fill=blue!20, circle,
	minimum height=3em]
\tikzstyle{state} = [draw, fill=blue!20, circle,
	minimum height=3em]

%***********************************************%
%                                               %
% Document										%
%                                               %
%***********************************************%

\begin{document}
\lab{4}

\section*{Note:}
\begin{itemize}
	\item The lab should be completed individually.
	\item The lab must be submitted by the end of Thursday office hours the
		following week and checked off by a GSI.
\end{itemize}

\section{Introduction}
You have been supplied with three, obfuscated, buggy ISR modules. These modules
should implement the ISR functionality from Project 2 correctly, however they 
are not quite right. The supplied ISRs are using a 4-cycle version of the 
pipelined multiplier. The module has the regular definition: 

\texttt{module ISR(reset, value, clock, result, done);}

\section{Assignment}
For this lab, you will need to write a testbench that will catch all the three
buggy modules. Your testbench must use a task to check the solution of the ISR
module (this is not that easy to do). See the testbenches from the previous labs
if you do not remember how to write a task.

In addition to catching the bugs, you must figure out what the source of each 
bug is. This will require you to think critically. Since you do not have access 
to the source of the module, you will have to look at its outputs and try to 
theorize what could be causing the errors you are seeing. This is not easy, you 
will have to think hard about how the ISR should work and how that compares to 
what you are seeing.

\emph{Hint:} The bugs in \texttt{buggy1} and \texttt{buggy3} are in the ISR 
module. The bug in \texttt{buggy2} is in one of the submodules.


\section{Submission}
Unlike previous labs, we cannot autograde your lab. You will need to show it to
one of the GSIs. In order to get your work checked off, you will need to type up 
explanations of the bugs and show the file to a GSI. Once you think you have 
identified and described the bugs, please add yourself to the help queue and 
mark that you wish to be checked off.

\end{document}
