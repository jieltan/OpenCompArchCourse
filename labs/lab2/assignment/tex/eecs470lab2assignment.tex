%***********************************************%
%												%
% EECS 470 - Lab 02								%
%<------------------> 							%
% Last Modified by:								%
%	William Cunningham on 2014-9-11				%
%												%
%***********************************************%

%***********************************************%
% Preamble										%
%***********************************************%

\documentclass{article}

\usepackage{
	tikz,
	xcolor,
	colortbl,
	graphicx,
	amsmath,
	amssymb,
	hyperref,
	mathrsfs,
	float,
	siunitx,
	fancyhdr,
	url,
	minted,
	cleveref
}

\usepackage
	[left=1in,top=1in,right=1in,bottom=1in]
	{geometry}

\pagestyle{fancy}
\raggedright

%--- Header ---%

\newcommand{\courseNumber}{EECS 470}
\newcommand{\courseTitle}{Computer Architecture}
\newcommand{\university}{University of Michigan, Ann Arbor}
\newcommand{\labdate}{January 28$^\text{th}$/29$^\text{th}$, 2021}


\lhead{
	\small{
		\university
	}
}
\rhead{
	\small{
		\emph{Date: \labdate} \hspace*{-1em}
		% Why is the above \hspace necessary?
	}
}

\newcommand{\shortbar}{
	\vspace*{-12pt}
	\begin{center}
		\rule{5ex}{0.1pt}
	\end{center}
}
\newcommand{\lab}[1]{
	\begin{center}
		\LARGE{
			\vspace*{-32pt}
			EECS 470 Lab #1 Assignment
			\shortbar
			\vspace*{-20pt}
		}
	\end{center}
}


%***********************************************%
%                                               %
% TikZ Definitions                              %
%                                               %
%***********************************************%

\usetikzlibrary{shapes,arrows}

% Block Diagram Styles
\tikzstyle{block} = [draw, fill=blue!20, rectangle,
        minimum height=3em, minimum width=6em]
\tikzstyle{sum} = [draw, fill=blue!20, circle, node distance=2cm]
\tikzstyle{input} = [coordinate]
\tikzstyle{output} = [coordinate]
\tikzstyle{branch} = [coordinate]
\tikzstyle{pinstyle} = [pin edge={to-, thin, black}]

% Signal Flow Graph Styles
\tikzstyle{signal} = [draw, fill=blue!20, circle,
	minimum height=3em]
\tikzstyle{state} = [draw, fill=blue!20, circle,
	minimum height=3em]

%***********************************************%
%                                               %
% Document										%
%                                               %
%***********************************************%

\begin{document}
\lab{2}

\section*{Note:}
\begin{itemize}
	\item The lab should be completed individually.
	\item The lab must checked off by a GSI before the end of next week Friday's lab.\end{itemize}

\section{Assignment}
\subsection{Writing a Robust Testbench}
Some modules are simple enough that you can write a testbench with complete
coverage. Often times, you will have to strike a balance between a thorough
testbench and the amount of time required to generate the testbench, or for the
testbench to run. We have supplied you with a slightly obfuscated full-adder 
(\texttt{fa1.v}) and a stub testbench (\texttt{fa1\_test.v}). You need to write
\texttt{fa1\_test.v} such that it achieves complete coverage of the full-adder 
module. Once you have done this, it should be a quick fix to identify and fix 
the bug in \texttt{fa1.v}.

\emph{Note:} to compile this lab, we need a Makefile, but we weren't provided with
one. Think about what you learned today in lab. Write your own Makefile (or 
modify the one provided in Lab 1). We'll also need a TCL script since that 
wasn�t provided either.

\subsection{Using Arrays of Modules}
Using the 1-bit adder from Part A, build a 64-bit adder as shown in the
presentation. The included file named \texttt{fa64.v} contains a module stub 
\texttt{full\_adder\_64bit}. Make the module into a complete 64 bit adder. (This
should take no more than 5 minutes, but if it does, please ask around for help 
and discuss with your neighbors.)

\subsection{Writing a Harder Testbench}
\label{sec:hard-testbench}
We have provided you with a slightly more complete \texttt{fa64\_test.v}. As a 
first step, read through the code we provided and make sure you understand it. You will need a
lot of this material for Lab 3.

Try running the testbench. What's happening? Why? Read the testbench and
understand what it's doing.

\emph{Hint:} Use \texttt{Ctrl-}$\mathtt{\backslash}$ to kill an unresponsive simulation.


Since our testbench won't work as written, what kind of tests could you add to
achieve good, but imperfect coverage? Modify \texttt{fa64\_test.v} to be a 
useful testbench and make sure your \texttt{fa64.v} implementation passes. 
Specifically, you need to add some specialized test cases. What specific test 
cases are useful and/or a good idea to validate your adder? Be prepared to 
defend your choices. Our \texttt{compare\_correct\_sum} task is incomplete. You
need to identify what it's missing and fix it.

\subsection{Synthesizing Your Design}
Synthesize your 64 bit adder. Ensure that your testbench from
\cref{sec:hard-testbench} also 
passes successfully on the synthesized version. This is the first time we've 
synthesized a project with multiple source files so we'll need to make a few 
changes:

\begin{itemize}
	\item In the TCL script: (1) \texttt{read\_file -f sverilog [list ``fa1.v''
		``fa64.v'']}
		
	and (2) \texttt{set CLK\_PERIOD 1}

	\item In the Makefile, the rule to make the \texttt{.vg} file needs to
		depend on all of the design files.
	\item The generated \texttt{.vg} file will match the design name set in the
		TCL script. This name must be the top level module in the design, in our
		case this is \texttt{full\_adder\_64bit}.
\end{itemize}

Once you have successfully run synthesis, look at the .rep file. Search for the
word ``slack''. It's been violated! What does this mean? How do you fix this 
problem?

\section{Submission}
Once you are confident in your module and testbench, go to office hours, place yourself on the \href{https://oh.eecs.umich.edu/courses/eecs470}{\underline{help queue}} and a GSI will check you off.
\end{document}
