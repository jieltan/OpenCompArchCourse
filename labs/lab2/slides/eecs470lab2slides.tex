%*******************************************************%
%														%
% William Cunningham									%
% wcunning@umich.edu									%
% EECS 470 -- Lab 2										%
%														%
%*******************************************************%

%*******************************************************%
% Preamble												%
%*******************************************************%

\documentclass[dvipsnames]{beamer}
\usetheme{Lab}

\usepackage{
	siunitx,
	tikz,
	graphicx,
	amsmath,
	float,
	minted,
	hyperref,
	textcomp,
	upquote
}

%-----------------------%
% TikZ					%
%-----------------------%
%--- CircuiTikZ Definitions ---%

%--- TikZ Definitions ---%
\usetikzlibrary{shapes,arrows,automata,shadows}
\pgfdeclarelayer{background}
\pgfdeclarelayer{foreground}
\pgfsetlayers{background,main,foreground}
% Block Diagram Styles
\tikzstyle{block} = [draw, fill=RoyalBlue!50, rounded corners, rectangle,
        minimum height=2cm, minimum width=2cm]
\tikzstyle{sum} = [draw, fill=blue!20, circle, node distance=2cm]
\tikzstyle{input} = [coordinate]
\tikzstyle{output} = [coordinate]
\tikzstyle{branch} = [coordinate]
\tikzstyle{pinstyle} = [pin edge={to-, thin, black}]
% Signal Flow Graph Styles
\tikzstyle{signal} = [draw, fill=blue!20, circle,
	minimum height=3em]

\definecolor{darkblue}{rgb}{0,0,0.8}

\hypersetup{colorlinks=true,linkcolor=,urlcolor=red}

%*******************************************************%
% Document												%
%*******************************************************%

\title[Lab 2: Build System]{EECS 470 Lab 2}
\subtitle{Synopsys Build System}
\institute[University of Michigan]{Department of Electrical Engineering and 
			Computer Science \\
			College of Engineering \\
			University of Michigan}
\date{January 28/29, 2021}

\begin{document}
\frame{\titlepage}

\begin{frame}{Overview}
	\tableofcontents
\end{frame}

\section{Administrivia}
\begin{frame}{Administrivia}

\begin{block}{Homework}
		\begin{itemize}
			\item Homework 1 is due Thursday, 4th$^{\text{th}}$ February, 2021 11:59 PM (turn in via Gradescope)
		\end{itemize}
\end{block}
\begin{block}{Projects}
	\begin{itemize}
        \item Project 1 is due Friday, 29th$^{\text{th}}$ January, 2021 11:59 PM (turn in via submission script)
		\item We will monitor Piazza up to the project deadline. Please create a private post if you are having trouble submitting.
	\end{itemize}
\end{block}
	\begin{block}{}
		We are available to answer your questions. Office hours can
		be found on the course page and questions can be posted in the course
		Piazza.
	\end{block}
\end{frame}

\begin{frame}{Administrivia}

\begin{block}{Labs}
		\begin{itemize}
			\item Lab 1 is due by the end of lab (12:30pm) on 1/29!
			\item Lab 2 is due by the end of lab (12:30pm) on 2/5!
		\end{itemize}
\end{block}
\end{frame}

\section{Generic Verilog Design}

\subsection{Preprocessor Directives}
\begin{frame}{Preprocessor Directives}
	\begin{block}{What is a preprocessor?}
		\begin{itemize}
			\item Prepares code/designs for compilation
				\begin{itemize}
					\item Combines included files
					\item Replaces comments and whitespace
				\end{itemize}
		\end{itemize}
	\end{block}
	\begin{block}{Why do I care?}
		\begin{itemize}
			\item Programmable
			\item Generic Designs (parameterized caches anyone?)
		\end{itemize}
	\end{block}
\end{frame}

\begin{frame}{Verilog Macros}
	\begin{block}{Definitions}
		\begin{itemize}
			\item Syntax: \texttt{\`{}define <NAME> <value>} 
				\begin{itemize}
					\item Note: That is a \textbf{tick}, not an apostrophe 
						before define
					\item The tick character is found with the \~{}, usually
						above tab (on US standard layout keyboards)
				\end{itemize}
			\item Usage: \texttt{\`{}<NAME>}
			\item Good for naming constants (can be used like wires)
			\item Convention dictates that macro names be in all caps
			\item Can also \texttt{\`{}undef <NAME>}
		\end{itemize}
	\end{block}
\end{frame}

\begin{frame}{Verilog Macros}
	\begin{block}{Flow Control}
		Text inside flow control is conditionally included in the
		compiled source file.
	\end{block}
	\begin{columns}
		\begin{column}[T]{0.9\textwidth}
			\begin{itemize}
				\item[\texttt{\`{}ifdef}] Checks if something is defined
				\item[\texttt{\`{}else}] Normal else behavior
				\item[\texttt{\`{}endif}] End the if
				\item[\texttt{\`{}ifndef}] Checks if something is not
					defined
			\end{itemize}
		\end{column}
	\end{columns}	
\end{frame}

\begin{frame}{Verilog Macros by Example}
	\vspace*{-12pt}
	\inputminted[frame=lines,fontsize=\scriptsize,tabsize=4,obeytabs=true]{verilog}{verilog/macro.v}
\end{frame}

\begin{frame}[fragile]{Verilog Headers}
	\begin{block}{What is inclusion?}
		\begin{itemize}
			\item Paste the code from the specified file where the directive is
				placed
		\end{itemize}
	\end{block}
	\begin{block}{Where will I use this?}
		\begin{itemize}
			\item System defines separated out, e.g.
				\begin{minted}[fontsize=\scriptsize]{verilog}
`define ALU_OP_ADD 5'b10101
				\end{minted}
				\begin{minted}[fontsize=\scriptsize]{verilog}
`define DCACHE_NUM_WAYS 4'b0100
				\end{minted}
			\item Macro assertion functions, printing functions, etc.
			\item Note: headers are named with the \texttt{.vh} extension
		\end{itemize}
	\end{block}
\end{frame}

\begin{frame}[fragile]{Verilog Headers}
	\begin{block}{Inclusion}
		\begin{itemize}
			\item Syntax: \texttt{\`{}include <FILE\_NAME>}
			\item Pastes the contents of \texttt{<FILE\_NAME>} wherever the include appears
		\end{itemize}
	\end{block}
	\begin{block}{Include Guards}
		\begin{itemize}
			\item Inside the header\dots
				\begin{minted}[fontsize=\scriptsize]{verilog}
`ifndef __FILE_NAME_H__
`define __FILE_NAME_H__
				\end{minted}
				$ \mathtt{\vdots} $ 
				\begin{minted}[fontsize=\scriptsize]{verilog}
`endif
				\end{minted}
		\end{itemize}
	\end{block}
	\begin{block}{Note}
		\texttt{\`{}include} does not work with the autograder and is not
		allowed for projects 1, 2 and 3.
	\end{block}
\end{frame}

\begin{frame}{Verilog Inclusion by Example}
	\begin{block}{\texttt{and4.v}}
		\inputminted[frame=lines,obeytabs,tabsize=4,fontsize=\scriptsize]{verilog}{verilog/and4.v}
	\end{block}
\end{frame}

\begin{frame}{Verilog Inclusion by Example}
	\begin{block}{\texttt{and2.v}}
		\inputminted[frame=lines,obeytabs,tabsize=4,fontsize=\scriptsize,firstline=3,lastline=9]{verilog}{verilog/and2.v}
	\end{block}
\end{frame}

\begin{frame}{Verilog Inclusion by Example}
	\begin{block}{Better \texttt{and2.v}}
		\inputminted[frame=lines,obeytabs,tabsize=4,fontsize=\scriptsize]{verilog}{verilog/and2.v}
	\end{block}
\end{frame}

\begin{frame}{More Information}
	\begin{block}{Bibliography}
		Much of this information was taken from the 
		\href{http://www.veripool.org/papers/Preproc_Good_Evil_SNUGBos10_paper.pdf}{Verilog Preprocessor}
		paper by Wilson Snyder of Cavium Networks. It is \emph{highly}
		recommended reading.
	\end{block}
\end{frame}

\subsection{Parameters}
\begin{frame}[fragile]{Parameters}
	\begin{block}{What is a \texttt{parameter}?}
		\begin{itemize}
			\item Constant defined inside a module
			\item Used to set module properties
			\item Can be overridden on instantiation
		\end{itemize}
	\end{block}
	\begin{block}{How do I use them?}
		\begin{itemize}
			\item Definition
				\begin{minted}[fontsize=\scriptsize]{verilog}
parameter NUM_CACHE_LINES = 8;
				\end{minted}
		\end{itemize}
	\end{block}
\end{frame}

\begin{frame}[fragile]{Setting Parameters}
	\begin{block}{Overriding}
		\begin{itemize}
			\item Set parameters for a module on instantiation
			\item Allows for different versions different places
			\item Usage:
				\begin{minted}[fontsize=\scriptsize]{verilog}
cache #(.NUM_CACHE_LINES(4)) d_cache(...);
				\end{minted}
		\end{itemize}
	\end{block}
	\begin{block}{\texttt{defparam}}
		\begin{itemize}
			\item Set parameters for a module 
			\item Usage: 
				\begin{minted}[fontsize=\scriptsize]{verilog}
defparam dcache.NUM_CACHE_LINES = 4;
				\end{minted}
		\end{itemize}
	\end{block}
\end{frame}

\begin{frame}{Macros Vs. Parameters}
	\begin{columns}
		\begin{column}[T]{0.45\textwidth}
			\begin{block}{Macros}
				\begin{itemize}
					\item Possibly globally scoped -- namespace collision
					\item Use for modules that can change, but only have one
						instance
					\item Particularly for caches and naming arbitrary constants
					\item Needs the \texttt{\`{}} in usage
				\end{itemize}
			\end{block}
		\end{column}
		\begin{column}[T]{0.1\textwidth}
			\vspace*{0.3\textheight}
			\Large{vs.}
			\normalsize
		\end{column}
		\begin{column}[T]{0.45\textwidth}
			\begin{block}{Parameters}
				\begin{itemize}
					\item Locally scoped -- no namespace collision
					\item Use for modules with many instances at different sizes
					\item Particularly for \texttt{generate} blocks (which are
						in a later lab)
					\item Does not need extra characters (like the
						\texttt{\`{}})
				\end{itemize}
			\end{block}
		\end{column}
	\end{columns}
\end{frame}

\subsection{Hierarchical Design}
\begin{frame}{Array Connections}
	\begin{itemize}
		\item Make a simple module a duplicate it several times
		\item Assume we have a module defintion: 
			\begin{itemize}
				\item \texttt{one\_bit\_addr(a, b, cin, sum, cout);}
			\end{itemize}
		\item All ports are 1 bit, the first three inputs, last two outputs
		\item How do we build an eight bit adder?
	\end{itemize}
\end{frame}

\begin{frame}[fragile]{The Error Prone Way}
\begin{minted}[fontsize=\scriptsize,frame=lines,obeytabs,tabsize=4]{verilog}
module eight_bit_addr(
    input en, cin,
    input [7:0] a, b,
    output [7:0] sum,
    output cout);

    wire [6:0] carries;

    one_bit_addr a0(en, a[0], b[0], cin, sum[0], carries[0]);
    one_bit_addr a1(en, a[1], b[1], carries[0], sum[1], carries[1]);
    one_bit_addr a2(en, a[2], b[2], carries[1], sum[2], carries[2]);
    one_bit_addr a3(en, a[3], b[3], carries[2], sum[3], carries[3]);
    one_bit_addr a4(en, a[4], b[4], carries[3], sum[4], carries[4]);
    one_bit_addr a5(en, a[5], b[5], carries[4], sum[5], carries[5]);
    one_bit_addr a6(en, a[6], b[6], carries[5], sum[6], carries[6]);
    one_bit_addr a7(en, a[7], b[7], carries[6], sum[7], cout);
endmodule
\end{minted}
\end{frame}

\begin{frame}{The Error Prone Way}
	\begin{itemize}
		\item Lots of duplicated code
		\item Really easy to make mistake
		\item Now try building a 64-bit adder..., 256?
		\item There is a one line substitute
	\end{itemize}
\end{frame}

\begin{frame}[fragile]{The Better Way}
\begin{minted}[fontsize=\scriptsize,frame=lines,obeytabs,tabsize=4]{verilog}
module eight_bit_addr(
    input en, cin,
    input [7:0] a, b,
    output [7:0] sum,
    output cout);

    wire [6:0] carries;

    one_bit_addr addr [7:0] (
        .en(en), .a(a), .b(b), .cin({carries,cin}),
        .sum(sum), .cout({cout,carries})
    );
endmodule
\end{minted}

All of the ports in the \texttt{one\_bit\_addr} module are 1 bit wide. All of the
busses we pass are 8 bits wide, so each instantiation of the module will get
one, except \texttt{en} which is only 1 bit wide and thus copied to every module.

\end{frame}

\begin{frame}[fragile]{The (Even) Better Way}
\begin{minted}[fontsize=\scriptsize,frame=lines,obeytabs,tabsize=4]{verilog}
`define ADDR_WIDTH 8
module eight_bit_addr(
    input en, cin,
    input [(`ADDR_WIDTH-1):0] a, b,
    output [(`ADDR_WIDTH-1):0] sum,
    output cout);

    wire [(`ADDR_WIDTH-2):0] carries;

    one_bit_addr addr [(`ADDR_WIDTH-1):0] (
        .en(en), .a(a),.b(b),.cin({carries,cin}),
        .sum(sum),.cout({cout,carries})
    );
endmodule
\end{minted}
\end{frame}

\begin{frame}[fragile]{Array Connections: Pitfalls and Errors}
What happens if a wire isn't 1 or N bits wide?
\begin{minted}[fontsize=\scriptsize,frame=lines,obeytabs,tabsize=4]{verilog}
wire foo;
wire [3:0] bar;
wire [2:0] baz;

simple s1 [3:0](.a(foo), .b(bar), .c(baz));
\end{minted}
\scriptsize{\texttt{test.v:14: error: Port expression width 3 does not match expected width 4
or 1}} \normalsize \\
But still easy to accidentally promote a wire to a bus
\end{frame}


\section{Tools}

\subsection{Make}

\begin{frame}{What is Make?}
	\begin{block}{What is Make?}
		\begin{itemize}
			\item Build System
				\begin{itemize}
					\item Automatically build an \emph{executable} from
						\emph{source} files
					\item Rules for building are stored in a \texttt{Makefile}
				\end{itemize}
			\item Essentially a script of compilation commands
			\item Handles dependency resolution
		\end{itemize}
	\end{block}
	\pause
	\begin{block}{Why do I care?}
		\pause
		\begin{itemize}
			\item \texttt{simv} is an \emph{executable}
			\item \emph{Verilog} files are \emph{source} files
			\item \texttt{vcs} is a compilation command
		\end{itemize}
	\end{block}
\end{frame}

\begin{frame}{Anatomy of a \texttt{Makefile}}
	\begin{columns}
		\begin{column}[c]{0.70\textwidth}
			\begin{itemize}
				\item[\texttt{targets}] are what we want to build
				\item[\texttt{dependencies}] are what we need to build it
				\item[\texttt{commands}] are how to build it 
			\end{itemize}
		\end{column}
	\end{columns}
	\begin{block}{}	
		This looks something like the following
		\inputminted[fontsize=\scriptsize,frame=lines]{makefile}{scripts/Makefile.example}
	\end{block}
\end{frame}

\begin{frame}{Lab 1 \texttt{Makefile}}
	\inputminted[fontsize=\scriptsize,frame=lines,firstline=33,lastline=34,fontsize=\scriptsize]{makefile}{scripts/Makefile} 
	\inputminted[fontsize=\scriptsize,frame=lines,firstline=41,lastline=45,fontsize=\scriptsize]{makefile}{scripts/Makefile}

	What do the \texttt{\$()} mean?
\end{frame}

\begin{frame}{\texttt{Makefile} Macros}
	A \emph{macro} is a way to store custom build behavior. 

	\inputminted[frame=lines,firstline=15,lastline=15,fontsize=\scriptsize]{makefile}{scripts/Makefile}
	\inputminted[frame=lines,firstline=25,lastline=26,fontsize=\scriptsize]{makefile}{scripts/Makefile}
	\inputminted[frame=lines,firstline=33,lastline=34,fontsize=\scriptsize]{makefile}{scripts/Makefile} 
\end{frame}

\begin{frame}{Dependency Resolution}
	\begin{block}{What does it mean to be a dependency?}
		\begin{itemize}
			\item Dependencies are used during compilation, e.g. list of source
				files
			\item Might be intermediate or primary 
		\end{itemize}
	\end{block}
	\begin{block}{How does Make resolve dependencies?}
		\begin{itemize}
			\item Does the target exist? 
				\begin{itemize}
					\item[No --] Build it.
					\item[Yes --] When was it built in relation to the
						dependencies?
						\begin{itemize}
							\item[After --] Stop.
							\item[Before --] Rebuild.
						\end{itemize}
				\end{itemize}
		\end{itemize}
	\end{block}
\end{frame}

\begin{frame}{Automatic \texttt{Makefile} Macros}
	\begin{itemize}
		\item[\texttt{\$@}] The Current Target Name
		\item[\texttt{\$\^}] A List of All Dependencies
		\item[\texttt{\$<}] The First Prerequisite
		\item[\texttt{\$?}] A List of All Dependencies Requiring Rebuilding
	\end{itemize}
\end{frame}

\begin{frame}{Special \texttt{Makefile} Targets}
	\begin{columns}
		\begin{column}[c]{0.75\textwidth}
			\begin{itemize}
				\item[\texttt{.DEFAULT}] Sets what runs when \texttt{make} is executed
					with no arguments 
				\item[\texttt{.PHONY}] Dependencies will be built
					unconditionally 
				\item[\texttt{.PRECIOUS}] Dependencies will be kept,
					intermediate or not
				\item[\texttt{.SECONDARY}] Dependencies are automatically
					treated as intermediates, but not deleted
			\end{itemize}
		\end{column}
	\end{columns}
\end{frame}

\begin{frame}{\texttt{Make} Resources}
	\begin{itemize}
		\item \href{http://www.gnu.org/software/make/manual/make.pdf}{GNU Make Manual}
			\begin{itemize}
				\item Special Targets
				\item Automatic Variables
				\item Makefile Conventions 
			\end{itemize}
		\item \href{http://en.wikipedia.org/wiki/Make_(software)}{Wikipedia - Make (Software)}
	\end{itemize}	
\end{frame}

\subsection{VCS}
\begin{frame}{Intro to VCS}
	\begin{block}{What is VCS?}
		\begin{itemize}
			\item Synopsys Verilog Compiler Simulator
			\item Builds Simulators from Verilog (structural or behavioral)
			\item We barely manage to brush the surface\dots
		\end{itemize}
	\end{block}
	\begin{block}{Why do we care?}
		\begin{itemize}
			\item Knowledge is power\dots
			\item \dots Specifically the power to debug
		\end{itemize}
	\end{block}
\end{frame}

\begin{frame}{VCS by Example}
	\inputminted[frame=lines,firstline=15,lastline=15,fontsize=\scriptsize]{makefile}{scripts/Makefile}
	\begin{columns}
		\begin{column}[c]{0.75\textwidth}
			\begin{itemize}
				\item[\texttt{SW\_VCS}] CAEN specific; sets which one of the
					installed VCS versions is run
				\item[\texttt{-sverilog}] Interpret designs using the
					SystemVerilog standard
				\item[\texttt{+vc}] Allow direct C code hooks in the design
				\item[\texttt{-Mupdate}] Compile incrementally
				\item[\texttt{-line}] Allow interactive debugging (might need
					other options)
				\item[\texttt{-full64}] Build 64 bit executables
			\end{itemize}
		\end{column}
	\end{columns}
\end{frame}

\begin{frame}{More VCS Options}
	\begin{columns}
		\begin{column}[c]{0.75\textwidth}
			\begin{itemize}
				\item[\texttt{+define}] Define a \texttt{\`{}define} macro \\ 
					Ex. +define+DEBUG, Ex. +define+CLK=10
				\item[\texttt{+lint=all}] Provide many more warnings
				\item[\texttt{-gui}] Build a DVE executable for interactive
					waveform debugging
				\item[\texttt{-o}] Name of the executable generated
				\item[\texttt{-R}] Run the executable after compilation
			\end{itemize}
		\end{column}
	\end{columns}
\end{frame}

\subsection{Design Compiler}
\begin{frame}{Intro to Design Compiler and Synthesis}
	\begin{block}{What is synthesis?}
		\begin{itemize}
			\item The process of turning Behavioral Verilog into Structural
				Verilog
			\item Using a technology library
		\end{itemize}
	\end{block}
	\begin{block}{Why do I care?}
		\begin{itemize}
			\item Example of how a design might be built
			\item Informed guesses about\dots 
				\begin{itemize}
					\item Power
					\item Area
					\item Performance
				\end{itemize}
		\end{itemize}
	\end{block}
\end{frame}

\begin{frame}{Intro to Design Compiler and Synthesis}
	\begin{block}{What is Design Compiler?}
		\begin{itemize}
			\item The Synopsys synthesis tool
			\item Industry-leading
		\end{itemize}
	\end{block}
	\begin{block}{Why do we care?}
		\begin{itemize}
			\item The tool for EECS 470
			\item Once again, because knowledge is power\dots
			\item \dots Specifically, the power to Google
		\end{itemize}
	\end{block}
\end{frame}

\begin{frame}{Design Compiler by Example}
	\inputminted[fontsize=\scriptsize,frame=lines,firstline=13,lastline=17]{tcl}{scripts/tut_synth.tcl}
	\begin{columns}
		\begin{column}[c]{0.75\textwidth}
			\begin{itemize}
				\item[\texttt{read\_file}]
					Read through the files in the list for dependency resolution
					and error, perform operator replacement
				\item[\texttt{design\_name}] 
					Design to be synthesized
				\item[\texttt{clock\_name}]
					Name of the clock net 
				\item[\texttt{reset\_name}]
					Name of the reset net
				\item[\texttt{clk\_period}]
					Clock period ($T_{clk}$) in ns
			\end{itemize}
		\end{column}
	\end{columns}
\end{frame}

\begin{frame}{Design Compiler by Example}
	\inputminted[fontsize=\scriptsize,frame=lines,firstline=119,lastline=119]{tcl}{scripts/tut_synth.tcl}
	\begin{columns}
		\begin{column}[c]{0.75\textwidth}
			\begin{itemize}
				\item[\texttt{compile}] This command actually synthesizes the
					design. \texttt{-map\_effort} can be medium or high.
			\end{itemize}
		\end{column}
	\end{columns}
	\inputminted[fontsize=\scriptsize,frame=lines,firstline=118,lastline=118]{tcl}{scripts/tut_synth.tcl}
	\begin{columns}
		\begin{column}[c]{0.75\textwidth}
			\begin{itemize}
				\item[\texttt{\$chk\_file}] This command creates the
					\texttt{*.chk} file with the design warnings
			\end{itemize}
		\end{column}
	\end{columns}
	\inputminted[fontsize=\scriptsize,frame=lines,firstline=120,lastline=120]{tcl}{scripts/tut_synth.tcl}
	\begin{columns}
		\begin{column}[c]{0.75\textwidth}
			\begin{itemize}
				\item[\texttt{\$netlist}] Writing with \texttt{-format} verilog
					generates the \texttt{*.vg} file containing the structural
					verilog output
			\end{itemize}
		\end{column}
	\end{columns}
\end{frame}
			
\begin{frame}{Design Compiler by Example}
	\inputminted[fontsize=\scriptsize,frame=lines,firstline=121,lastline=121]{tcl}{scripts/tut_synth.tcl}
	\begin{columns}
		\begin{column}[c]{0.75\textwidth}
			\begin{itemize}
				\item[\texttt{\$ddc\_file}] This command generates the
					\texttt{*.ddc} file, which is included in designs where we
					want to synthesize a module separately. More on this in 
					Project 2. This replaces the \texttt{*.db} file mentioned in
					Lab 1.
			\end{itemize}
		\end{column}
	\end{columns}
\end{frame}

\begin{frame}{Design Compiler by Example}
	\inputminted[fontsize=\scriptsize,frame=lines,firstline=122,lastline=125]{tcl}{scripts/tut_synth.tcl}
	\begin{columns}
		\begin{column}[c]{0.75\textwidth}
			\begin{itemize}
				\item[\texttt{\$rep\_file}] This command generates the
					\texttt{*.rep} file, containing the timing report for the
					synthesized design. More on this in Project 2.
			\end{itemize}
		\end{column}
	\end{columns}
\end{frame}

\begin{frame}{Design Compiler by Example}
	\inputminted[fontsize=\scriptsize,frame=lines,firstline=44,lastline=45]{makefile}{scripts/Makefile}
	\begin{columns}
		\begin{column}[c]{0.75\textwidth}
			\begin{itemize}
				\item[$|$ tee] This command saves the output of the synthesis
					tool into a file. Errors and warnings found in this file are
					particularly important.
			\end{itemize}
		\end{column}
	\end{columns}
\end{frame}

\begin{frame}{Designs in Memory}
	\begin{columns}
		\begin{column}[T]{0.45\textwidth}
			\begin{block}{\texttt{analyze\dots elaborate}}
				\begin{itemize}
					\item Creates intermediate files 
						\begin{itemize}
							\item Specifically, templates of generated designs with
								parameterized sizes
						\end{itemize}
					\item Automatically links design components to templates
					\item Specifically resolves several errors that you will
						encounter in the final project
			\end{itemize}
			\end{block}
		\end{column}
		\begin{column}[T]{0.1\textwidth}
			\vspace*{0.3\textheight}
			\Large{vs.}
			\normalsize
		\end{column}
		\begin{column}[T]{0.45\textwidth}
			\begin{block}{\texttt{read\_file}}
				\begin{itemize}
					\item Simpler, single command
					\item No intermediate files
					\item No linking
					\item Can cause problems
				\end{itemize}
			\end{block}
		\end{column}
	\end{columns}
\end{frame}

\begin{frame}{Lab Assignment}
	\begin{itemize}
		\item Assignment is posted on the  
			\href{https://www.eecs.umich.edu/courses/eecs470/?page=schedule.php}{
			course web site}.
		\item If you get stuck\dots
			\begin{itemize}
				\item Ask us in Office Hours!
				\item Ask on Piazza
			\end{itemize}
		\item When you are finished, put yourself on \href{https://oh.eecs.umich.edu/courses/eecs470}{help queue} to get checked off
		\item Check-off is due by end of lab (12:30pm) on 2/5
	\end{itemize}
\end{frame}
\end{document}
