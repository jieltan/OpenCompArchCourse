%*******************************************************%
%														%
% William Cunningham									%
% wcunning@umich.edu									%
% EECS 470 -- Lab 5										%
%														%
%*******************************************************%

%*******************************************************%
% Preamble													%
%*******************************************************%

\documentclass[table,dvipsnames]{beamer}
\usetheme{Lab}

\usepackage{
	siunitx,
	tikz,
	graphicx,
	amsmath,
	float,
	minted,
	hyperref,
	textcomp,
	upquote,
	multirow,
	fancyvrb
}

\usepackage[T1]{fontenc}
%-----------------------%
% TikZ					%
%-----------------------%
%--- CircuiTikZ Definitions ---%

%--- TikZ Definitions ---%
\usetikzlibrary{shapes,arrows,automata,shadows,decorations,fadings}
\pgfdeclarelayer{background}
\pgfdeclarelayer{foreground}
\pgfsetlayers{background,main,foreground}
\tikzstyle{master-blank} = 
[	draw, 
	rounded corners, 
	rectangle,
	color=ForestGreen!25,
	minimum height=0.5cm, 
	minimum width=1.30cm
]
\tikzstyle{master-commit} = 
[	draw, 
	fill=ForestGreen!50, 
	rounded corners, 
	rectangle,
	minimum height=0.5cm, 
	minimum width=1.30cm
]
\tikzstyle{branch-commit} = 
[	draw, 
	fill=RoyalBlue!70, 
	rounded corners,
	rectangle,
	minimum height=0.5cm, 
	minimum width=1.30cm
]
\tikzstyle{clean-repo} = 
[	draw, 
	inner color=RoyalBlue!40, 
	outer color=RoyalBlue!50, 
	color=black,
	circle,
	minimum height=2cm
]
\tikzstyle{changes-repo} = 
[	draw, 
	inner color=Red!40, 
	outer color=Red!50, 
	color=black,
	circle,
	minimum height=2cm
]

\definecolor{darkblue}{rgb}{0,0,0.8}

\hypersetup{colorlinks=true,linkcolor=,urlcolor=red}

\newcommand{\masterrepo}[2]{
	\scriptsize{\color{Black}\texttt{#1}} \\
	\scriptsize{\color{OliveGreen}\texttt{HEAD=#2}}
}

\newcommand{\branchrepo}[2]{
	\scriptsize{\color{Black}\texttt{#1}} \\
	\scriptsize{\color{NavyBlue}\texttt{HEAD=#2}}
}

\newcommand{\commit}[1]{
	\scriptsize{\texttt{#1}}
}

%*******************************************************%
% Document												%
%*******************************************************%

\title[Lab 5: Scripting]{EECS 470 Lab 5}
\subtitle{Linux Shell Scripting}
\institute[University of Michigan]{Department of Electrical Engineering and 
			Computer Science \\
			College of Engineering \\
			University of Michigan}
\date{Thursday, 3$^{\text{rd}}$ October, 2019}

\begin{document}
\frame{\titlepage}

\begin{frame}{Overview}
	\tableofcontents
\end{frame}

\section{Administrivia}
\begin{frame}{Administrivia}
	\begin{block}{Homework}
		\begin{itemize}
			\item Homework 3 is due Friday, 11$^{\text{st}}$ October at midnight
		\end{itemize}
	\end{block}
	\begin{block}{Projects}
		\begin{itemize}
			\item Project 3 is due Monday, 7$^{\text{th}}$ October at midnight
			\item If you haven't, you need to get started \emph{now}
		\end{itemize}
	\end{block}
	\begin{block}{}
		We are available to answer questions on anything here. Office hours can
		be found in the
		\href{https://www.eecs.umich.edu/courses/eecs470/?page=home.php}{course
		website}.
	\end{block}
\end{frame}

\section{UNIX}
\begin{frame}{UNIX}
	\begin{block}{What is UNIX?}
		\begin{itemize}
			\item Mainframe operating system
			\item Written by Dennis Ritchie, Ken Thompson and Rob Pike (among
				others) at Bell Labs
			\item The basis for many modern operating systems, e.g. Linux, BSD,
				Mac OSX
		\end{itemize}
	\end{block}
	\begin{block}{History of UNIX}
		\begin{itemize}
			\item Written at Bell Labs in 1969
			\item First version of BSD is installed in 1974
			\item Last Bell Labs UNIX (Version 7) is published in 1979
			\item The GNU Project is started by Richard Stallman
			\item Linux Torvalds writes a monolithic kernel operating system in
				1991
		\end{itemize}
	\end{block}
\end{frame}

\begin{frame}{UNIX Philosophy}
	\begin{block}{}
		``This is the Unix philosophy: Write programs that do one thing and do 
		it well. Write programs to work together. Write programs to handle text
		streams, because that is a universal interface.''

		\hfill -- Douglas McIlroy
	\end{block}
	\begin{block}{}
		``Everything is a file descriptor or a process.''

		\hfill -- Linus Torvalds
	\end{block}
\end{frame}

\subsection{Files}
\begin{frame}{Files}
	\begin{block}{What is a \emph{file}?}
		\begin{itemize}
			\item Anything referenced through a filesystem
			\item Anything with a file descriptor
		\end{itemize}
	\end{block}
\end{frame}

\begin{frame}{Types of Files}
	\begin{itemize}
		\item Regular -- Anything not in one of the following categories
		\item Directory -- Can contain other files and directories (read up on
			inodes sometime)
		\item Symbolic Link -- A pointer to another file
		\item Pipe -- Covered in a few slides 
		\item Socket -- Covered in EECS 482/489
	\end{itemize}
\end{frame}

\begin{frame}{Permissions}
	\centering
	\begin{columns}
		\begin{column}[T]{0.5\textwidth}
		\begin{table}
			{\tt
			\begin{tabular}{|l|l|l|l|l|l|l|l|l|}
				\hline
				%--[ user  ]--%
				\cellcolor{ForestGreen!25}r& 
				\cellcolor{ForestGreen!25}w& 
				\cellcolor{ForestGreen!25}x&
				%--[ group ]--% 
				\cellcolor{BlueGreen!25}-&
				\cellcolor{BlueGreen!25}-&
				\cellcolor{BlueGreen!25}-&
				%--[ other ]--% 
				\cellcolor{RoyalBlue!25}-&
				\cellcolor{RoyalBlue!25}-&
				\cellcolor{RoyalBlue!25}-\\ \hline
				%--[ description ]--%
				\multicolumn{3}{|c|}{\cellcolor{ForestGreen!25}User} & 
				\multicolumn{3}{c|}{\cellcolor{BlueGreen!25}Group} & 
				\multicolumn{3}{c|}{\cellcolor{RoyalBlue!25}Other} \\ \hline
			\end{tabular}
			}
			\caption{UNIX permissions are represented with one bit for each
				permission in each category, e.g. \texttt{700}.}
		\end{table}
		\end{column}
	\end{columns}
	\begin{columns}
		\begin{column}[T]{0.3\textwidth}
			\begin{block}{Three Permissions}
				\begin{itemize}
					\item Read 
					\item Write 
					\item Execute
				\end{itemize}
			\end{block}
		\end{column}
		\hspace*{-1cm}
		%\begin{column}[T]{0.15\textwidth}
			%\vspace*{6pt}
			%for each of
		%\end{column}
		\hspace*{-1cm}
		\begin{column}[T]{0.3\textwidth}
			\begin{block}{Three Categories}
				\begin{itemize}
					\item User
					\item Group 
					\item Other
				\end{itemize}
			\end{block}
		\end{column}
	\end{columns}
\end{frame}


\subsection{Utilities}
\begin{frame}{Utilities}
	\begin{block}{What is a \emph{utility}?}
		\begin{itemize}
			\item A program used to process text streams/files
			\item Called from some command line/shell
		\end{itemize}
	\end{block}
	\begin{block}{Why do I care?}
		\begin{itemize}
			\item Utilities form the basis of ``Linux skills''
			\item Useful for automation
			\item Necessary for today's lab
		\end{itemize}
	\end{block}
\end{frame}


\begin{frame}{Navigation Utilities}
	\begin{block}{\texttt{pwd}}
		\begin{itemize}
			\item Description: print working directory -- where you are
			\item Synopsis: \texttt{pwd [OPTIONS]}
		\end{itemize}
	\end{block}
	\begin{block}{\texttt{ls}}
		\begin{itemize}
			\item Description: list directory contents
			\item Synopsis: \texttt{ls [OPTIONS]\dots~ [FILE] \dots}
		\end{itemize}
	\end{block}
	\begin{block}{\texttt{cd}}
		\begin{itemize}
			\item Description: change directory
			\item Synopsis: \texttt{cd [OPTIONS] [PATH]}
		\end{itemize}
	\end{block}
\end{frame}

\begin{frame}{File Utilities}
	\begin{block}{\texttt{cp}}
		\begin{itemize}
			\item Description: copy files
			\item Synopsis: \texttt{cp [OPTONS]\dots~ SOURCE DEST}
		\end{itemize}
	\end{block}
	\begin{block}{\texttt{mv}}
		\begin{itemize}
			\item Description: move files
			\item Synopsis: \texttt{mv [OPTONS]\dots~ SOURCE DEST}
		\end{itemize}
	\end{block}
	\begin{block}{\texttt{rm}}
		\begin{itemize}
			\item Description: remove files
			\item Synopsis: \texttt{rm [OPTONS]\dots~ FILE\dots}
		\end{itemize}
	\end{block}
\end{frame}

\begin{frame}{\texttt{diff}}
	\begin{block}{Description}
		\begin{itemize}
			\item Shows the line-by-line differences between files
			\item Good for checking if your output is correct
		\end{itemize}
	\end{block}
	\begin{block}{Synopsys}
		\begin{itemize}
			\item \texttt{diff [OPTIONS] FILES}
		\end{itemize}
	\end{block}
	\begin{block}{Examples}
		\begin{itemize}
			\item \texttt{diff -uy ../project3\_correct/writeback.out
				writeback.out}
			\item \texttt{vimdiff ../project3\_correct/writeback.out
				writeback.out}
		\end{itemize}
	\end{block}
\end{frame}

\begin{frame}{Graphical Diff}
	\begin{block}{Alternatives to \texttt{diff}}
		Parsing output of \texttt{diff} is hard, so it might be useful to use 
		some kind of ``graphical diff'' tool, like \texttt{vimdiff}, which opens
		up two files side-by-side in Vim showing their differences. Try it and
		you'll see how much easier to parse this.
	\end{block}
\end{frame}

\begin{frame}{\texttt{grep}}
	\begin{block}{Description}
		\begin{itemize}
			\item Print lines matching a pattern
		\end{itemize}
	\end{block}
	\begin{block}{Synopsys}
		\begin{itemize}
			\item \texttt{grep [OPTIONS] PATTERN [FILE]}
		\end{itemize}
	\end{block}
	\begin{block}{Examples}
		\begin{itemize}
			\item \texttt{grep '@@@' program.out}
			\item \texttt{ps -axfuw | grep "\$USER" | grep "vcs"}
		\end{itemize}
	\end{block}
\end{frame}

\begin{frame}[fragile]{Regular Expressions}
	\begin{itemize}
		\item Really powerful/useful
		\item Complicated, and beyond the scope of this presentation
		\item Read up on them
			\begin{itemize}
				\item at
					\href{http://en.wikipedia.org/wiki/Regular_expressions}{Wikipedia}
				\item in the
					\href{http://www.gnu.org/software/grep/manual/grep.html#Regular-Expressions}{\texttt{grep} manual}
			\end{itemize}
	\end{itemize}
\end{frame}

\begin{frame}[fragile]{\texttt{man}}
	\begin{block}{Description}
		\begin{itemize}
			\item An interface to the on-line reference manuals (pager) 
		\end{itemize}
	\end{block}
	\begin{block}{Synopsys}
		\begin{itemize}
			\item \texttt{man PAGE}
		\end{itemize}
	\end{block}
	\begin{block}{Examples}
		\begin{itemize}
			\item \href{http://www.gnu.org/software/grep/manual/grep.html}{\texttt{man grep}}
			\item
				\href{http://www.gnu.org/software/diffutils/manual/diffutils.html}{\texttt{man
				diff}}
		\end{itemize}
	\end{block}
\end{frame}

\begin{frame}{Pager}
	\begin{block}{Definition}
		\begin{itemize}
			\item A program which allows browsing of large text files by
				breaking them into screen-sized chunks. 
		\end{itemize}
	\end{block}
	\begin{block}{Examples}
		\begin{itemize}
			\item \texttt{less}
			\item \texttt{man}
		\end{itemize}
	\end{block}
\end{frame}

\begin{frame}{Other Utilities}
	\begin{block}{These will be useful\dots}
		\begin{columns}
			\begin{column}[T]{0.45\textwidth}
				\begin{itemize}
					\item \texttt{cut}
					\item \texttt{touch}
					\item \texttt{tee}
					\item \texttt{xargs}
				\end{itemize}
			\end{column}
			\begin{column}[T]{0.45\textwidth}
				\begin{itemize}
					\item \texttt{tail}
					\item \texttt{column}
					\item \texttt{find}
					\item \texttt{less}
				\end{itemize}
			\end{column}
		\end{columns}
	\end{block}
	\begin{block}{These are harder, but even more useful\dots}
		\begin{columns}
			\begin{column}[T]{0.45\textwidth}
				\begin{itemize}
					\item \texttt{sed}
					\item \texttt{awk}
					\item \texttt{patch}
					\item \texttt{vi(m)}
				\end{itemize}
			\end{column}
			\begin{column}[T]{0.45\textwidth}
				\begin{itemize}
					\item \texttt{fmt}
					\item \texttt{tmux} 
					%\item 
					%\item 
				\end{itemize}
			\end{column}
		\end{columns}
	\end{block}
\end{frame}

\subsection{Connecting Utilities}

\begin{frame}{Program Features}
	\begin{block}{Methods of Communication}
		\begin{itemize}
			\item Standard Text Streams
				\begin{itemize}
					\item \texttt{stdin} 
					\item \texttt{stdout}
					\item \texttt{stderr}
				\end{itemize}
			\item Return value/code
		\end{itemize}
	\end{block}
\end{frame}

\begin{frame}{Return Codes}
	\begin{block}{What is a \emph{return code}?}
		\begin{itemize}
			\item The integer value a program returns (e.g. \texttt{return(0);})
			\item Conventionally, returning zero indicates success, non-zero
				failure
			\item Specific values other than zero mean different things for
				different programs
		\end{itemize}
	\end{block}
\end{frame}

\begin{frame}{Standard Text Streams}
	\begin{block}{\texttt{stdin}}
		\begin{itemize}
			\item The default input to a program
			\item From a keyboard by default
		\end{itemize}
	\end{block}
	\begin{block}{\texttt{stdout}}
		\begin{itemize}
			\item The default output of a program
			\item To a display by default
		\end{itemize}
	\end{block}
	\begin{block}{\texttt{stderr}}
		\begin{itemize}
			\item The default error output of a program
			\item To a display by default, requires additional effort to save
		\end{itemize}
	\end{block}
\end{frame}


\begin{frame}{Connecting Utilities}
	\begin{block}{Why do we want to connect utilities?}
		\begin{itemize}
			\item Combination jobs without intermediate files
			\item e.g. take the diff of two different grep operations (what you
				need to do for today's lab)
		\end{itemize}
	\end{block}
	\begin{block}{How can we connect utilities?}
		\begin{itemize}
			\item Pipes
			\item Redirection
		\end{itemize}
	\end{block}
\end{frame}

\begin{frame}{Pipe}
	\begin{block}{What is a \emph{pipe}?}
		\begin{itemize}
			\item Connects the \texttt{stdout} of one program to the
				\texttt{stdin} of another
			\item Does not connect \texttt{stderr}
		\end{itemize}
	\end{block}
	\begin{block}{How do I use one?}
		\begin{itemize}
			\item Call a program on one side of the \texttt{|} and then call
				another on the other side
			\item e.g. \texttt{dmesg | less}
		\end{itemize}
	\end{block}
\end{frame}

\begin{frame}{Pipe: \texttt{xargs}}
	\begin{block}{Problem}
		What if we want to pipe to utility that uses arguments instead of input?
	\end{block}
	\begin{block}{Solution: \texttt{xargs}}
		\href{http://en.wikipedia.org/wiki/Xargs}{\texttt{xargs}} splits input 
		into individual items and calls the program that is its argument once 
		for each input.
	\end{block}
\end{frame}

\begin{frame}{Redirection}
	\begin{block}{What is \emph{redirection}?}
		\begin{itemize}
			\item Allows for modification of the standard text streams
				\begin{itemize}
					\item \texttt{stdin} is \texttt{0}
					\item \texttt{stdout} is \texttt{1}
					\item \texttt{stderr} is \texttt{2}
				\end{itemize}
			\item Several types:
				\begin{itemize}
					\item[\texttt{0<}] Use a file instead of the keyboard for
						\texttt{stdin}
					\item[\texttt{i>}] Use a file instead of the terminal for
						the stream \texttt{i}
					\item[\texttt{i>{}>}] Like \texttt{i>}, but append to a file 
						instead of overwriting
					\item[\texttt{i>\&j}] Put stream \texttt{i} into the same
						place as stream \texttt{j}
				\end{itemize}
		\end{itemize}
	\end{block}
	\href{http://www.tldp.org/LDP/abs/html/io-redirection.html}{Advanced Bash Scripting Guide: I/O Redirection}
\end{frame}

\begin{frame}{Redirection by Example}
	\begin{block}{Example}
		\begin{itemize}
			\item \texttt{./vs-asm < test\_progs/evens.s > program.mem}
		\end{itemize}
	\end{block}
	\begin{block}{Example}
		\begin{itemize}
			\item \texttt{make | tee 2>\&1 build.out}
		\end{itemize}
	\end{block}
	\begin{block}{Example}
		\begin{itemize}
			\item \texttt{./test 1>\&2 2>\&3}
		\end{itemize}
	\end{block}
\end{frame}

\section{Bourne Again Shell}

\begin{frame}{Shell}
	\begin{block}{What is a \emph{shell}?}
		\begin{itemize}
			\item Before we had graphical environments, we had text shells
			\item Basically, an interpreter for commands, executing programs
				and saving information
			\item Possibly a Read-Execute-Print-Loop (REPL)
		\end{itemize}
	\end{block}
\end{frame}

\begin{frame}{Bourne Again Shell}
	\begin{block}{What is \emph{BASH}?}
		\begin{itemize}
			\item Stands for the Bourne Again Shell
			\item Created in 1989 by Brian Fox
			\item Default shell in most Linux distributions and Mac OSX
		\end{itemize}
	\end{block}
	\begin{block}{Why BASH?}
		\begin{itemize}
			\item What I learned first
			\item Default in CAEN Redhat, finally
		\end{itemize}
	\end{block}
\end{frame}

\begin{frame}{Warning}
	\begin{block}{Warning}
		Everything after this slide will be specific to BASH. Other shells
		behave similarly, but not identically. If you want to use something
		else (e.g. ZSH, TCSH, etc.), please find other resources.
	\end{block}
\end{frame}

\subsection{Variables}

\begin{frame}{Variables}
	\begin{block}{BASH Variables}
		\begin{itemize}
			\item Store data
			\item Contain text, for the most part
			\item No type system
		\end{itemize}
	\end{block}
	\begin{block}{Syntax}
		\begin{itemize}
			\item Assignment/Declaration
				\begin{itemize}
					\item \texttt{variable=value}
				\end{itemize}
			\item Referencing
				\begin{itemize}
					\item \texttt{\$variable}
					\item \texttt{\$\{variable\}}
				\end{itemize}
		\end{itemize}
	\end{block}
\end{frame}

\begin{frame}{Variable Scope}
	\begin{block}{Scope}
		\begin{itemize}
			\item Variables exist inside the shell they're in
			\item Unless exported \\
				e.g. \texttt{export EDITOR=vim}
			\item This is very important in scripts, particularly the shell
				startup scripts (\texttt{.bashrc}, \texttt{.bash\_profile})
		\end{itemize}
	\end{block}
\end{frame}

\begin{frame}{Special Variables}
	\begin{columns}
		\begin{column}[T]{0.9\textwidth}
			\begin{itemize}
				\item[\texttt{\$\#}] The number of command line arguments
				\item[\texttt{\$0}] The name of the script/function called
				\item[\texttt{\$1}] The first argument to the script/function
				\item[\texttt{\$?}] The return code of the last program run in this
					shell
				\item[\texttt{\$USER}] The current user
				\item[\texttt{\$HOME}] The user's home directory
			\end{itemize}
		\end{column}
	\end{columns}
\end{frame}

\subsection{Flow Control}

\begin{frame}[fragile]{Flow Control}
	\begin{block}{\texttt{if/else}}
		\begin{minted}[fontsize=\scriptsize,frame=lines]{bash}
			if [ var -eq "string" ]
			then
				command
			elif [ var -eq "string2" ] 
			then
				command2
			else
				command3
			fi
		\end{minted}
	\end{block}
\end{frame}

\begin{frame}[fragile]{Flow Control}
	\begin{block}{case}
		\begin{minted}[fontsize=\scriptsize,frame=lines]{bash}
			case "$var" in
				val)
					command
				;;
				val2)
					command
				;;
				( * )
					default
				;;
			esac
		\end{minted}
	\end{block}
\end{frame}

\begin{frame}[fragile]{Flow Control}
	\begin{block}{\texttt{for}}
		\begin{minted}[fontsize=\scriptsize,frame=lines]{bash}
			for file in ./*; do
				command $file
				command2
			done
		\end{minted}
		\begin{minted}[fontsize=\scriptsize,frame=lines]{bash}
			for (( a=1; a <= LIMIT; a++ )) do
				command 
				command2
			done
		\end{minted}
	\end{block}
\end{frame}

\begin{frame}{Conditionals}
	\begin{block}{Testing}
		\begin{itemize}
			\item Testing happens in \texttt{[ ]}
			\item String tests are different than arithmetic tests
			\item Generally use \texttt{-lt, -gt, -le, -ge, -eq, -ne}
			\item Tests for files are special, e.g. \texttt{[ -x simv ]} makes
				sure that the binary \texttt{simv} has execution permissions
		\end{itemize}
	\end{block}
\end{frame}

\subsection{Functions}

\begin{frame}[fragile]{Functions}
	\begin{block}{Functions}
		\begin{itemize}
			\item Packages of commands
			\item Arguments are referenced by position
			\item Useful for packaging up commonly reused bits
		\end{itemize}
	\end{block}
	\begin{block}{Syntax}
		\begin{minted}[fontsize=\scriptsize,frame=lines]{bash}
			function () 
			{
			    commands
			}
		\end{minted}
	\end{block}
\end{frame}

\subsection{Globbing}

\begin{frame}{File Globbing}
	\begin{block}{What is \emph{globbing}?}
		\begin{itemize}
			\item File name wildcarding in the shell
			\item Expansion is done, and then passed to the command to be
				executed
			\item e.g. \texttt{test\_progs/*.s}
		\end{itemize}
	\end{block}
	\begin{block}{What should I know globbing?}
		\begin{itemize}
			\item Superior to parsing \texttt{ls} output in everyway
			\item Can get more complicated, see
				\href{http://www.gnu.org/software/bash/manual/html_node/Pattern-Matching.html\#Pattern-Matching}{this
				page} of the BASH manual.
		\end{itemize}
	\end{block}
\end{frame}

\section{Scripting}

\begin{frame}{Scripting}
	\begin{enumerate}
		\item Write a series of shell commands into a text file
		\item On the first line of the file, specify an interpreter \\
			e.g. \texttt{\#!/bin/bash}
		\item Name it something appropriate \\
			e.g. \texttt{test.sh}
		\item Add execute permissions \\
			e.g. \texttt{chmod +x test.sh}
		\item Run the script \\
			e.g. \texttt{\$ ./test.sh}
	\end{enumerate}
\end{frame}

\section{Assignment}
\begin{frame}[fragile]{Lab Assignment}
	\begin{itemize}
		\item Assignment is posted to the course website as 
			\href{http://www.eecs.umich.edu/eecs/courses/eecs470/labs/eecs470lab5assignment.pdf}{Lab
			5 Assignment}.
		\item If you get stuck\dots
			\begin{itemize}
				\item Ask a neighbor, quietly
				\item Put yourself in the 
					\href{https://oh.eecs.umich.edu/course_queues/403}{help queue}
			\end{itemize}
		\item When you finish the assignment, sign up in the
			\href{https://oh.eecs.umich.edu/course_queues/403}{help queue}
			and mark that you would like to be checked off.
		\item If you are unable to finish today, the assignment needs to be
			checked off by a GSI in office hours \emph{before} the next lab
			session.
	\end{itemize}
\end{frame}

\end{document}
