%***********************************************%
%												%
% EECS 470 - Lab 04								%
%<------------------> 							%
% Last Modified by:								%
%	William Cunningham on 2013-12-25			%
%												%
%***********************************************%

%***********************************************%
% Preamble										%
%***********************************************%

\documentclass[dvipsnames]{article}

\usepackage{
	tikz,
	xcolor,
	colortbl,
	graphicx,
	amsmath,
	amssymb,
	float,
	siunitx,
	fancyhdr,
	hyperref,
	cleveref,
	minted
}

\usepackage
	[left=1in,top=1in,right=1in,bottom=1in]
	{geometry}

\pagestyle{fancy}
\raggedright

%--- Header ---%

\newcommand{\courseNumber}{EECS 470}
\newcommand{\courseTitle}{Computer Architecture}
\newcommand{\university}{University of Michigan, Ann Arbor}
\newcommand{\labdate}{3$^{\text{rd}}$ October, 2019}


\lhead{
	\small{
		\university
	}
}
\rhead{
	\small{
		\emph{Date: \labdate} \hspace*{-1em}
		% Why is the above \hspace necessary?
	}
}

\newcommand{\shortbar}{
	\vspace*{-12pt}
	\begin{center}
		\rule{5ex}{0.1pt}
	\end{center}
}
\newcommand{\lab}[1]{
	\begin{center}
		\LARGE{
			\vspace*{-32pt}
			EECS 470 Lab #1 Assignment
			\shortbar
			\vspace*{-20pt}
		}
	\end{center}
}


%***********************************************%
%                                               %
% TikZ Definitions                              %
%                                               %
%***********************************************%

\usetikzlibrary{shapes,arrows}

% Block Diagram Styles
\tikzstyle{block} = [draw, fill=blue!20, rectangle,
        minimum height=3em, minimum width=6em]
\tikzstyle{sum} = [draw, fill=blue!20, circle, node distance=2cm]
\tikzstyle{input} = [coordinate]
\tikzstyle{output} = [coordinate]
\tikzstyle{branch} = [coordinate]
\tikzstyle{pinstyle} = [pin edge={to-, thin, black}]

% Signal Flow Graph Styles
\tikzstyle{signal} = [draw, fill=blue!20, circle,
	minimum height=3em]
\tikzstyle{state} = [draw, fill=blue!20, circle,
	minimum height=3em]

%***********************************************%
%                                               %
% Document										%
%                                               %
%***********************************************%

\begin{document}
\lab{5}

\section*{Note:}
\begin{itemize}
	\item The lab should be completed individually.
	\item The lab must be submitted by the end of Friday the
		following week and checked off by a GSI.
\end{itemize}

\section{Introduction}
You have just learned about the BASH shell and some of the useful utilities
provided by the standard GNU/Linux environment. Today you will be using this
knowledge to automate the testing of Project 3. The general form of this script
will closely resemble the actual Project 3 Autograder, though there are a number
of additional testcases which are not released to you.

\section{Assignment}
You are required to automatically check that a modified version of the
Verisimple pipeline produces correct output. The following parts lay out how we 
recommend you do this, but you are free to do this however you would like.

\subsection{Ground Truth}
\label{sec:truth}
First, we need a \emph{ground truth}, a set of known-correct outputs. To build
this, you should write a simple script to loop through files in a directory (the
assembly testcases) and call a  command or commands on each of them. This should
look something like:

\inputminted[frame=lines]{bash}{scripts/ground-truth-skeleton.sh}

\subsection{Testing}
Now that you have a script that generates some ground truth data, it is time to
modify it to compare that data with the output of some unknown version of the
Verisimple pipeline. This will be done in several steps.

First, you will need to run the script you wrote in \cref{sec:truth} to generate
your ground truth. Then, you will need to make sure the executable you are 
running is up-to-date. After that, for each of your testcases you will need to
\begin{itemize}
	\item Simulate the testcase. Do you need to rebuild the executable
		for this?
	\item Check that the \texttt{writeback.out} files match exactly with
		the ground truth version. What utility does this?
	\item Check that the lines beginning with \texttt{@@@} in the 
		\texttt{program.out} files match exactly. What utility lets us
		find only these lines, before using the one from above to
		compare?
	\item Print whether the testcase passed or failed.
\end{itemize}

\section{Check Off}
In order to get your work checked off, you will need to show your \textbf{well-commented}
scripts to a GSI. Once you think you have your scripts working, please add 
yourself to the help queue and mark that you wish to be checked off. While you
wait, you might consider implementing one of the optional features listed below
or working on the project itself.

\textbf{In addition, you also need to show your git repository of project 3 and grant Admin access to the account \texttt{eecs470staff@umich.edu}.}

\appendix

\section{Optional Features}
This was a very basic version of this script, and we can improve it
significantly from here forward. Here are a few things you could do to make this
script better/prettier/more useful for the final project\dots

\begin{itemize}
	\item Use BASH functions to modularize this script. I would recommend adding
		this to your \texttt{.bash\_profile} so that you can call (e.g.
		\texttt{source .test.bash}).
	\item Colorize your output. I find it more satisfying to read something like
		\textcolor{Green}{\texttt{PASSED}} than to read \texttt{PASSED} without the
		color. The Linux Documentation Project has a good description of 
		\href{http://www.tldp.org/HOWTO/Bash-Prompt-HOWTO/x329.html}{BASH colors}.
	\item You may want to move some significant portion of the handling of the
		testcases/assembly files into the Makefile/test bench. For instance, the 
		\texttt{\$readmemh} command in the test bench can be pointed at a file
		provided by a \texttt{\`{}define}, which in turn can be provided by an
		environment variable/Makefile macro. The dependency resolution
		capabilities of make is also very useful for making sure that you are
		running the right test case.
	\item Detecting and killing infinite loops or hung simulations is extremely
		useful, but also very hard. 
\end{itemize}

\end{document}
